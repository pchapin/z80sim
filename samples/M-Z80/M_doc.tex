%%%%%%%%%%%%%%%%%%%%%%%%%%%%%%%%%%%%%%%%%%%%%%%%%%%%%%%%%%%%%%%%%%%%%%%%%%%
% FILE   : m.tex
% AUTHOR : Peter Chapin
% SUBJECT: User-mode documentation for the M multi-tasking kernel.
%
% Send comments to Peter Chapin (PChapin@vtc.vsc.edu).
%%%%%%%%%%%%%%%%%%%%%%%%%%%%%%%%%%%%%%%%%%%%%%%%%%%%%%%%%%%%%%%%%%%%%%%%%%%

% ++++++++++++++++++++++++++++++++
% Preamble and global declarations
% ++++++++++++++++++++++++++++++++
\documentclass{article}
% \pagestyle{headings}
\setlength{\parindent}{0em}
\setlength{\parskip}{1.75ex plus0.5ex minus0.5ex}

% +++++++++++++++++++
% The document itself
% +++++++++++++++++++
\begin{document}

% ----------------------
% Title page information
% ----------------------
\title{The M System\\
       User-Mode Documentation}
\author{Peter Chapin}
\date{February, 1996}
\maketitle

\section{Introduction}

The M system is a simple multi-tasking kernel for an embedded controller. The current
implementation of M supports exactly four tasks (no more, no less) executing in infinite loops
out of ROM. M is implemented for a Z80 microprocessor.

Although simple, M is nevertheless a useful teaching tool. I present it in the operating systems
class at Vermont Technical College as a way of showing the students how some of the nitty-gritty
details of multi-processing might be done.

\section{Memory Map}

Broadly speaking, the memory map is divided into a ROM section and a RAM section. The ROM
section is divided into a data segment and a text (code) segment. The data segment is further
divided into a section for the system's use and sections for use by each of the four application
programs. The code segment is also divided into a system section and four application sections.
The system section is divided into the initialization code, the interrupt service routines, and
the system service calls. Each application is free to organize its own structure as it sees fit.
In particular, each application can define it's own subroutines. M does not support device
drivers in the usual sense. All low level hardware support is built directly into the kernel.

The RAM section is also divided into a section for the system and four application specific
sections. The application RAM contains a stack for each application. The system also maintains
its own stack although, in the current implementation, it isn't really necessary.

Due to the limitations of the assembler we used, every label in the entire system must have a
unique label. This means that applications cannot really choose their labels freely. There can
be no label name conflicts between the applications or between the applications and the system.
For a small system like M created by a small group of people, this isn't really that big a
problem.

\section{System Service Calls}

This section documents the service calls offered to the applications by the system.

\subsection{Terminal Handling}

\subsubsection{Get\_Char}

This routine returns a keystroke from the keyboard. If there are no keystrokes waiting in the
keyboard buffer, this routine blocks until one is available. The calling task should lock the
keyboard before calling this routine. If the calling task has not locked the keyboard, this
routine returns at once with a key code of zero.

\subsubsection{Put\_Char}

This routine prints a character to the console's display. It does not return until the character
has been printed. This implementation does not supported buffered output.

Multiple tasks can call Put\_Char at the same time. The order in which their output appears is
undefined, but all characters will be displayed without any loss of information. If one or more
tasks wish to synchronize their outputs, those tasks will have to cooperate between themselves.

\subsubsection{Lock\_Keyboard}

This routine uses semaphores internally to provide a task with exclusive access to the keyboard.
If another task has locked the keyboard when a certain task calls this routine, then the calling
task will be blocked until the task that owns the keyboard calls UnLock\_Keyboard. Be sure a
task does not attempt to lock the keyboard twice in a row. If a task attempts to do so, it will
deadlock.

\subsubsection{UnLock\_Keyboard}

This routine undoes the effect of Lock\_Keyboard. If the calling task has not locked the
keyboard, this routine has no effect. If other tasks are blocked trying to lock the keyboard,
this routine will unblock the highest priority task and keep the keyboard locked (the lock being
owned by the newly unblocked task).

\subsection{Process Control}

\subsubsection{Sema\_Down}

This routine performs a ``down'' operation on a semaphore. If the semaphore is not zero, it is
decremented. If the semaphore is zero, the calling task is blocked until some other task
performs a Sema\_Up call on the semaphore. The testing, decrementing, and blocking are done in
an atomic fashion.

\subsubsection{Sema\_Up}

This routine performs an ``up'' operation on a semaphore. If the semaphore is not zero, it is
incremented. If the semaphore is zero, this routine will unblock the highest priority task that
is waiting on that semaphore. In that case, the semaphore will remain at zero (the newly
unblocked task having effectively completed its ``down'' operation). If there are no tasks
waiting on the zero semaphore, then the semaphore is simply incremented.

\subsubsection{Set\_Priority}

This routine sets the priority of the calling task. The M scheduler runs the highest priority
unblocked task.

\section{Conclusion}

Pretty neat system!

\appendix

\section{To-do List}

There are many areas where M needs help. Here are some things to do with a few implementation
notes.

\subsection{Implement Priorities}

To implement task priorities the task table entry structure will need an additional priority
member. The initialization code will need to set the priority of each task to it's initial
value. I suggest an initial value of 10 to allow for both increases and decreases.

The scheduler will then have to be modified so that it schedules the highest priority unblocked
task. This means that it will have to search the entire task table at each interrupt. This needs
to happen in such a way as to allow tasks with the same priority to run in round-robin fashion.

\subsection{Implement the Idle Loop}

The system should have an idle loop to soak up otherwise unused CPU cycles. Once priorities have
been implemented, the idle loop could simply be a fifth, internal task that is initialized at a
priority of zero. The body of the task should just be a jump instruction that jumps to itself.

\subsection{Implement a Blocking Get\_Char}

The current Get\_Char busy waits if there are no characters in the keyboard buffer. In
particular, it sits in a loop checking the buffer to see if there are any characters in it. This
is horribly inefficient. Instead, Get\_Char should mark the calling task as blocked and then
schedule a different task to run if there are no characters in the keyboard buffer.

Doing this correctly is tricky. For one thing, you want to be sure that it checks the buffer and
marks the task as blocked in an atomic way. You don't want an untimely arrival of a keyboard
interrupt to cause problems. Scheduling another task is also tricky. The scheduler ISR is
designed to be called as an interrupt service routine. Is it possible to call it as a normal
routine? That question needs to be looked at.

In addition, the serial port ISR needs to be modified to mark the task as unblocked. Should the
keyboard ISR attempt to reschedule? (What if the task waiting for the keystroke is higher in
priority than the task that was executing at the time of the interrupt?)

\subsection{Implement Lock\_Keyboard}

One easy way to do this is to create an internal, system semaphore associated with the keyboard.
The Lock\_Keyboard function can then just do a Sema\_Down on that semaphore. UnLock\_Keyboard
can do the corresponding Sema\_Up. The only tricky part is the requirement that Get\_Char be
able to tell if the calling task has the keyboard locked or not. This could be done by checking
the Blocked flag and the Sema\_Address in the task table

\end{document}
