%%%%%%%%%%%%%%%%%%%%%%%%%%%%%%%%%%%%%%%%%%%%%%%%%%%%%%%%%%%%%%%%%%%%%%%%%%%
% FILE         : lab11.tex
% AUTHOR       : Peter Chapin
% LAST REVISED : November 1995
% SUBJECT      : EL-214, Interrupt Lab.
%
%
% Send comments to Peter Chapin (pchapin@night.vtc.vsc.edu) or at the
%   snail mail address of:
%
%       Peter Chapin
%       Vermont Technical College
%       Main Street
%       Randolph Center, VT. 05061
%
%%%%%%%%%%%%%%%%%%%%%%%%%%%%%%%%%%%%%%%%%%%%%%%%%%%%%%%%%%%%%%%%%%%%%%%%%%%

% ++++++++++++++++++++++++++++++++
% Preamble and global declarations
% ++++++++++++++++++++++++++++++++
\documentstyle{article}
% \pagestyle{headings}
\setlength{\parindent}{0em}
\setlength{\parskip}{1.75ex plus0.5ex minus0.5ex}

% +++++++++++++++++++
% The document itself
% +++++++++++++++++++
\begin{document}

\centerline{\Large{EL-214 Lab}}
\centerline{\Large{Interrupts}}
\vspace{0.5in}

\underline{\large{Introduction}}

Interrupts are essential in almost all non-trivial computer systems.
Whenever a single processor is trying to do several things at once,
interrupts are almost certainly involved. This is true for both large,
multi-user operating systems and small, dedicated hardware controllers.
Interrupts also allow the machine to work on CPU intensive tasks while
waiting for slow I/O devices (like keyboards and disk drives). Instead of,
for example, just sitting in a loop polling a switch, the switch can be
configured to generate an interrupt when it is active and thus free the
processor so it can tend to other activities while the switch is inactive.

In this lab, you will explore some of these issues.

\underline{\large{The Main Program}}

The main program in this lab simply counts on the seven segment display. It
includes a short delay so that you can follow the counting more easily.

\begin{verbatim}
DELAY   EQU     0FFDh
HEXTO7  EQU     0FF1h
SCAN    EQU     0FEEh

        ORG     1800h
        JP      Start

        ; Data used by this program.
Count:  DW      0000h

        ; ---------------
        ; Initialization.
        ; ---------------
Start:
        ; Initialize the 8255.
        LD      A, 80h
        OUT     (0C3h), A

        ; Store address of ISR into 1F41h.
        LD      A, LOW(ISR)
        LD      (1F41h), A
        LD      A, HIGH(ISR)
        LD      (1F42h), A

        ; Initialize the count.
        LD      BC, 00h
        LD      (Count), BC

        ; We can now accept interrupts.
        IM      1
        EI

        ; -------------
        ; Main Program.
        ; -------------
Loop:
        ; Convert the current count for display.
        LD      BC, (Count)
        LD      A, B
        LD      (1F12h), A
        LD      A, C
        LD      (1F13h), A
        LD      A, 00h
        LD      (1F14h), A
        CALL    HEXTO7

        ; Display it for 10 mS.
        CALL    SCAN

        ; Next count.
        LD      BC, (Count)
        INC     BC
        LD      (Count), BC
        JP      Loop

        ; ------------------------------
        ; The Interrupt Service Routine.
        ; ------------------------------

ISR:    ; You write this part!
        EI
        RET
\end{verbatim}

\underline{\large{Part 1}}

Build the hardware shown on the attached schematic. Connect the clock input
of the flip flop to the active high output of PB0. Thus whenever you press
PB0, you will generate an interrupt request.

Write an interrupt service routine that displays the current value of
`Count' on the 16 LEDs on the left side of the trainer (at ports 0C1h and
0C0h). With the program running, you should be able to push the button and
``capture'' the value of the counter at that moment.

Set the logic analyzer to trigger on the interrupt acknowledge machine
cycle. Capture a sample that shows a significant amount of information both
before and after the trigger event. Verify that the system is executing
your interrupt service routine. Record the instructions executed by the
CAMI before your ISR is actually entered.

\begin{enumerate}

\item Calculate how long the ISR takes to execute. (For the CAMI, T=558.7 nS)
Take into account the T states used by the instructions in the CAMI ROM.
Does the ISR represent a significant load on the system?

\item What is the processor spending most of its time doing?

\item What would happen if you connected the switch directly
to the $\overline{\mbox{INT}}$ line of the processor? Try it.

\end{enumerate}

\underline{\large{Part 2}}

Adjust a function generator to produce a TTL compatible square wave at it's
output (voltage from zero to five volts). Set the frequency to
approximately 1 Hz. Connect the output of the function generator to the
clock input of the flip flop in your interrupt circuit. Observe the effect
as you modify the frequency.

Modify your interrupt service routine so that it toggles the speaker bit on
the 8255 on the right hand side of the trainer. Set the function generator
to 100 Hz so that 100 interrupts are generated every second. Note the
effect on the speaker.

Using the 'scope, trigger on the square wave being fed to the speaker and
then observe the waveform from the function generator. Also observe the
interrupt acknowledge signal. Sketch these waveforms.

Increase the interrupt frequency until the main program stops. What is that
frequency? Observe what happens to the scanning as the fraction of time
spent in the interrupt service routine increases.

\begin{enumerate}

\item What is the frequency of the signal driving the speaker when the
interrupt frequency is 100 Hz?

\item You should be able to calculate the interrupt frequency at which
the main program stops making progress. How does it compare with your
observed frequency?

\end{enumerate}

\underline{\large{Write-Up}}

Submit listings of the programs you used, a description of your
observations, the waveform sketches you made in part 2 (with an
explaination of their meaning), and the answers to the questions.

\end{document}
