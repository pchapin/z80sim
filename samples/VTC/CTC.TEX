%%%%%%%%%%%%%%%%%%%%%%%%%%%%%%%%%%%%%%%%%%%%%%%%%%%%%%%%%%%%%%%%%%%%%%%%%%%
% FILE         : ctc.tex
% AUTHOR       : Peter Chapin
% LAST REVISED : November 1995
% SUBJECT      : CTC worksheet.
%
%
% Send comments to Peter Chapin (pchapin@night.vtc.vsc.edu) or at the
%   snail mail address of:
%
%       Peter Chapin
%       Vermont Technical College
%       Main Street
%       Randolph Center, VT. 05061
%
%%%%%%%%%%%%%%%%%%%%%%%%%%%%%%%%%%%%%%%%%%%%%%%%%%%%%%%%%%%%%%%%%%%%%%%%%%%

% ++++++++++++++++++++++++++++++++
% Preamble and global declarations
% ++++++++++++++++++++++++++++++++
\documentstyle{article}
% \pagestyle{headings}
\setlength{\parindent}{0em}
\setlength{\parskip}{1.75ex plus0.5ex minus0.5ex}

% +++++++++++++++++++
% The document itself
% +++++++++++++++++++
\begin{document}

\centerline{\Large{EL-214: Worksheet}}
\centerline{\Large{CTC}}
\vspace{0.5in}

\begin{enumerate}

\item Suppose you were working with a Z-80 CTC which was decoded so that
channels zero through three appeared at port addresses F0H through F3H.
Suppose also that you wanted to use channel zero to count 100 external
events. Write a program fragment which will send the correct bytes to the
CTC to configure it properly. Have the CTC interrupt the Z-80 when it is
finished counting and have it return the vector number 70H during the
interrupt acknowledge. (Make all don't care or unspecified bits in the
configuration byte zero).

\item Suppose the following program is executed to configure a Z-80 CTC
system. (The CTC's channel zero is at port address 20H).

\begin{verbatim}
LD   A,33H
LD   (2B40H),A
LD   A,6CH
LD   (2B41H),A
LD   A,2BH
LD   I,A

LD   A,40H
OUT  (20H),A
LD   A,A5H
OUT  (20H),A
LD   A,10H
OUT  (20H),A

IM   2
EI
\end{verbatim}

Answer the following two questions

\begin{enumerate}

\item The CTC is configured to interrupt the processor when it is done. What is
the address of the interrupt service routine?

\item The CTC is configured for timer mode, how long will it time given a 2 MHz
clock?

\end{enumerate}

\item Answer the following questions about Z-80 CTC configuration
limits. In all cases, assume that the Z-80 system clock
frequency is 4 MHz.

\begin{enumerate}

\item What is the maximum amount of time that can be
programmed in timer mode using only one channel?

\item What is the maximum amount of time that can be
programmed in timer mode using all four channels as timers? How should the
CTC be wired? When programing such an arrangement, should all four timers
be autotriggered?

\item What is the maximum amount of time that can be
programmed using all four channels in whatever way you desire? How should
the CTC be wired?

\item What is the maximum count possible using all four
channels in counter mode? How should the CTC be wired?

\end{enumerate}

\item Assume that the system you are working with contains a CTC.
The CTC is decoded so that channel zero is at port address 20H. Suppose you
wanted channel zero's output to produce pulses with a frequency of 250 Hz.
Show the instructions that you would use to configure the CTC. Make any
``don't care'' bits in the configuration byte(s) zero. Assume the system
clock rate is 4 MHz. No interrupts.

\end{enumerate}

\end{document}
