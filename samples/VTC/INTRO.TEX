%%%%%%%%%%%%%%%%%%%%%%%%%%%%%%%%%%%%%%%%%%%%%%%%%%%%%%%%%%%%%%%%%%%%%%%%%%%
% FILE         : intro.tex
% AUTHOR       : Peter Chapin
% LAST REVISED : August 1995
% SUBJECT      : General policies for EL-214 (Fall '95 semester).
%
%
% Send comments to Peter Chapin (pchapin@night.vtc.vsc.edu) or at the
%   snail mail address of:
%
%       Peter Chapin
%       Vermont Technical College
%       Main Street
%       Randolph Center, VT. 05061
%
%%%%%%%%%%%%%%%%%%%%%%%%%%%%%%%%%%%%%%%%%%%%%%%%%%%%%%%%%%%%%%%%%%%%%%%%%%%

% ++++++++++++++++++++++++++++++++
% Preamble and global declarations
% ++++++++++++++++++++++++++++++++
\documentstyle{article}
% \pagestyle{headings}
\setlength{\parindent}{0em}
\setlength{\parskip}{1.75ex plus0.5ex minus0.5ex}

% +++++++++++++++++++
% The document itself
% +++++++++++++++++++
\begin{document}

\centerline{\Large{EL-214: Microprocessor Techniques}}
\centerline{\Large{Introductory Information}}
\vspace{0.5in}

\underline{Instructor}:

My name is Peter Chapin. I can be reached in my office of G-109 at 728-1304
(I have voice mail) or by email at {\tt pchapin@night.vtc.vsc.edu}. In
times of extreme need, you can also try to contact me at home at 728-9669
(I have an answering machine there too).

You can arrange a time to meet with me if you contact me first. However, I
am either in my office or in G-111 much of the time, and I invite you to
stop by whenever you have a question. My class schedule is posted outside
my office.

\underline{Text}:

``The Z80 Microprocessor,'' second edition by Ramesh Gaonkar. There are
also other Z-80 books in the library, and I encourage you to supplement the
text by reading in some of them. Most of the books in the library pertain
to programming the Z-80.

\underline{Network Resources}:

I will use the S:$\backslash$ET$\backslash$EL-214 directory on the Novell
network to distribute sample programs, documentation files, and other
materials.

In addition, I may put some materials on the World Wide Web at the URL of

\begin{verbatim}
        http://lunchtime.vtc.vsc.edu/el-214/index.htm.
\end{verbatim}

\underline{Other Materials}:

You will need some high density disks to back up your programs. {\em Be
sure to back up your work regularly and carefully!} Even if you loose your
work because of a disk or network failure, I may not give you any
accomodation and your grade is likely to suffer.

\underline{Grading Policy}:

\begin{tabular}{|l|l|l|l|} \hline
         & How Many? & Points Each & Total Points            \\ \hline \hline
Labs     & about 13  & Weighted as 20\% of the total grade & \\ \hline
Quizzes  & about 10  & 10          & 100                     \\ \hline
Exams    & 3         & 100         & 300                     \\ \hline
Final    & 1         & 200         & 200                     \\ \hline
\end{tabular}

There will be a grand total of about 600 points depending on exactly how
many quizzes there are. Your grade will depend on the overall fraction of
those points you actually get. The lab grade (as provided by your lab
instructor) will be blended into your overall grade as 20\% of the total.
For example, if there really are 600 total points, then the lab would be
150 points giving a grand total of 750 points. If your lab grade was 80\%,
then you would get 120 points for your lab work.

All exams and quizzes are open notes and open book. I will drop your lowest
quiz grade (for example if I give nine quizzes only eight will become part
of the grading scheme). There will be no make-up quizzes, but I will find a
way to arrange for a make-up for a missed exam. If you know ahead of time
that you will miss an exam or quiz, let me know as soon as you can. Policy
regarding labs will be set by your lab instructor.

\underline{Objectives}:

You will learn the basics about microprocessor system architechure in this
course. This includes both the digital hardware and the assembly language
programming. Although we will focus on the Z-80 processor, what you learn
in this course will apply in a general way to many other processors. It
should be easy for you to learn about new and more advanced processor
systems once you've taken this course.

{\em Good luck in EL-214, and have fun!!}

\end{document}
